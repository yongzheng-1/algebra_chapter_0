\section{Chapter II.\hspace{0.2em} Groups, first encounter}

\subsection{\textsection1. Definition of group}

%Problem 1.1
\begin{problem}[1.1]
$\vartriangleright$ Write a careful proof that every group is the group of isomorphisms of a
groupoid. In particular, every group is the group of automorphisms of some object
in some category.[\S 2.1]
\end{problem}

\begin{solution}

\end{solution}

%Problem 1.2
\begin{problem}[1.2]
$\vartriangleright$ Consider the `sets of numbers' listed in \S 1.1, and decide which are made into
groups by conventional operations such as $+$ and $\cdot$. Even if the answer is negative
(for example, $(R, \cdot)$ is not a group), see if variations on the definition of these sets
lead to groups (for example, $(R^{\ast}, \cdot)$ is a group; cf. \S 1.4). [\S 1.2]
\end{problem}

\begin{solution}
\begin{enumerate}
\item $\mathbb{N}$ is not a group under operation $+$ or $\cdot$.
\item $\mathbb{Z}$ is a group under operation $+$, but not under operation $\cdot$.
\item $\mathbb{N}^{\ast}$ and $\mathbb{Z}^{\ast}$ can't be a group under operation $\cdot$.
\item $\mathbb{Q}, \mathbb{R}$ and $\mathbb{C}$ are groups under operation $+$, but can't be groups under operation $\cdot$.
\item $\mathbb{Q}^{\ast}, \mathbb{R}^{\ast}$ and $\mathbb{C}^{\ast}$ are groups under operation $\cdot$.
\end{enumerate}
These statements are easy to be verified. 
\end{solution}

% Problem 1.3
\begin{problem}[1.3]
Prove that $(gh)^{-1}=h^{-1}g^{-1}$ for all elements $g,h$ of a group $G$.
\end{problem}

\begin{solution}
Since $$(gh)(h^{-1}g^{-1}) = g(hh^{-1})g^{-1} = gg^{-1} = e_G$$ and $$(h^{-1}g^{-1})(gh) = h^{-1}(g^{-1}g)h=h^{-1}h=e_G,$$
it turns out $(gh)^{-1}=h^{-1}g^{-1}$.
\end{solution}

% Problem 1.4
\begin{problem}[1.4]
Suppose that $g^2 = e$ for all elements $g$ of a group $G$; prove that $G$ is commutative.
\end{problem}

\begin{solution}
By assumption, for all elements $g,h\in G$
$$hghg=e=gg$$
it means $hgh=g$ then $hghh=gh$. It implies $hg=gh$.i.e. $G$ is commutative.
\end{solution}

% Problem 1.5
\begin{problem}[1.5]
  The `multiplication table' of a group is an array compiling the results of all
  multiplications $g \bullet h$:
\begin{center}
\begin{tabular}{c||c|c|c|c}

  $\bullet$ & $e$ & $\cdots$ & $h$ & $\cdots$ \\ \hline \hline
  $e$ & $e$ & $\cdots$ & $h$ & $\cdots$ \\ \hline
  $\cdots$ & $\cdots$ & $\cdots$ & $\cdots$ & $\cdots$ \\ \hline
  $g$ & $g$ & $\cdots$ & $g\bullet h$ & $\cdots$ \\ \hline
  $\cdots$ & $\cdots$ & $\cdots$ & $\cdots$ & $\cdots$ \\
\end{tabular}
\end{center}
  (Here $e$ is the identity element. Of course the table depends on the order in
  which the elements are listed in the top row and leftmost column.) Prove that
  every row and every column of the multiplication table of a group contains all
  elements of the group exactly once (like Sudoku diagrams!).
\end{problem}

\begin{solution}
If $f\bullet g = f \bullet h$, then $g=h$. It is a contradiction.Hence there is no equal elements in $f$-row. 
If $g\bullet f= h \bullet f$, then $g=h$. It is a contradiction too. Hence there is no equal elements in $f$-column. 
\end{solution}

%Problem 1.6
\begin{problem}[1.6]
$\neg$ Prove that there is only one possible multiplication table for $G$ if $G$ has
exactly $1$, $2$, or $3$ elements. Analyze the possible multiplication tables for groups
with exactly $4$ elements, and show that there are two distinct tables, up to reordering
the elements of $G$. Use these tables to prove that all groups with $\leq 4$ elements are commutative.

(You are welcome to analyze groups with $5$ elements using the same technique,
but you will soon know enough about groups to be able to avoid such brute-force
approaches.) [2.19]
\end{problem}

\begin{solution}
If $\abs{G}=1$, then $G=\{1\}$. It is trivial group.

\noindent If $G=\{1, a\}$, then $a^2=1$. 

\noindent If $G=\{1, a, b\}$, then $ab=1$. It implies $a^2=b$. Hence $G$ is a cyclic group $\{1, a, a^2\}$. 

\noindent Now, let consider $G=\{1,a,b,c\}$
	
	If one of $a^2, b^2, c^2$ is not $1$, say, $a^2\neq 1$, then $a^3\neq 1$ and $G=\{1,a,a^2, a^3\}$ is a cyclic group. In fact, suppose $a^3=1$. 
	The possible values of $ab$ are $1$ or $c$. If $ab=1$, then $b=a^2$ and $ac=1$ because $ac=b$ implies $c=a$, this contradicts to our assumption. But if $ac=1$, $ab=1$, then $b=c$ also contradicts to our assumption. 
Therefore, we must have $ab=c, ac=b$. This leads to $a^2=1$ and contradicts to our assumption. Hence $a^3\neq 1$.

If $a^2=b^2=c^2=1$, then $G=\{1, a, b, ab\}$.
\end{solution}

% Problem 1.7
\begin{problem}[1.7]
Prove Corollary 1.11.
\end{problem}

\begin{solution}
($\implies$): This is Lemma 1.10.\\
($\impliedby$): Suppose $N=n\abs{g}$, then $g^N = (g^{\abs{g}})^n = e^n=e$.
\end{solution}

% Problem 1.8
\begin{problem}[1.8]
$\neg$ Let $G$ be a finite abelian group with exactly one element $f$ of order $2$. Prove
that $\prod_{g\in G}g=f$.[4.16]
\end{problem}

\begin{solution}
By assumption, there are only two elements $1, f$ such that $1^{-1} = 1, f^{-1} = f$. Hence 
$$\prod_{g\in G}g = (\prod_{g\in G\setminus \{1,f\}}g) \bullet f = (\underbrace{1\bullet 1\bullet \cdots 
\bullet 1}_{\frac{\abs{G} -2}{2}}) \bullet f = f.$$
The statement is proved.
\end{solution}

% Problem 1.9
\begin{problem}[1.9]
Let $G$ be a finite group, of order $n$, and let $m$ be the number of elements $g \in G$
of order exactly $2$. Prove that $n-m$ is odd. Deduce that if $n$ is even, then
$G$ necessarily contains elements of order $2$.
\end{problem}

\begin{solution}
Let's rewrite set $G$ as
$$G = \{1\} \cup \{g\big{|} g\in G, \abs{g} = 2\}\cup \{g\big{|} g\in G, \abs{g} > 2\}.$$ 
If $\abs{g} > 2$, then $\abs{g^{-1}} > 2$ and $g\neq g^{-1}$. Hence $\abs{\{g\big{|} g\in G, \abs{g} > 2\}} = 2k,k\in \mathbb{N}$ and 
$n=1+m+2k$. This is $n-m=2k+1$, an odd number. 

If $n$ is even number, then $m = n - (2k+1) > 0$, i.e. $G$ contain elements of order $2$. 
\end{solution}

% Problem 1.10
\begin{problem}[1.10]
Suppose the order of $g$ is odd. What can you say about the order of $g^2$?
\end{problem}

\begin{solution}
If $\abs{g}$ is odd, then $\gcd(2,\abs{g}) = 1$. Hence $\abs{g^2} = \frac{\abs{g}}{\gcd(2,\abs{g})} = \abs{g}$
\end{solution}

% Problem 1.11
\begin{problem}[1.11]
Prove that for all $g,h$ in a group $G$, $\abs{gh}=\abs{hg}$.(Hint: Prove that $\abs{aga^{-1}}=\abs{g}$ for all $a,g$ in G.)
\end{problem}

\begin{solution}
For $h,g\in G$, 
$$(hgh^{-1})^{\abs{g}} = hg^{\abs{g}}h^{-1} = 1.$$ Therefore $\abs{hgh^{-1}}\mid \abs{g}$.
On the other hand, $g=h^{-1}(hgh^{-1})h$ implies $\abs{g} \mid \abs{hgh^{-1}}$.
This proves $\abs{hgh^{-1}}=\abs{g}$. Taking $g=gh$, we have 
$\abs{hg} = \abs{hghh^{-1}}=\abs{gh}$
\end{solution}

% Problem 1.12
\begin{problem}[1.12]
$\vartriangleright$In the group of invertible $2\times 2$ matrices, consider
$$g=\left(\begin{array}{cc} 0 & -1 \\ 1 & 0 \end{array} \right), \indent h=\left(\begin{array}{cc} 0 & 1 \\ -1 & -1 \end{array} \right).$$
Verify that $\abs{g}=4$, $\abs{h}=3$ and $\abs{gh}=\infty$.[\S 1.6]
\end{problem}

\begin{solution}
$g \neq I, g^2=\left(\begin{array}{cc} -1 & 0 \\ 0 & -1 \end{array} \right)\neq I, 
g^3=\left(\begin{array}{cc} 0 & 1 \\ -1 & 0 \end{array} \right)\neq I, g^4 = I$.
So $\abs{g}=4$.\\
$h\neq I, h^2=\left(\begin{array}{cc} -1 & -1 \\ 1 & 0 \end{array} \right)\neq I, h^3 = I$, So $\abs{h} = 3$\\
$gh=\left(\begin{array}{cc} 1 & 1 \\ 0 & 1 \end{array} \right), (gh)^n = \left(\begin{array}{cc} 1 & n \\ 0 & 1 \end{array} \right)$
So. $\abs{gh} = \infty$.
\end{solution}

% Problem 1.13
\begin{problem}[1.13]
$\vartriangleright$Give an example showing that $\abs{gh}$ is not necessarily equal to $\lcm(\abs{g},\abs{h})$, even if $g$ and $h$
commute.[\S 1.6, 1.14]
\end{problem}

\begin{solution}
Let's consider the group $\mathbb{Z}_{12}$ defined as the following:
\begin{enumerate}
\item $\mathbb{Z}_{12} = \{[0], [1], [2],\cdots, [11]\}$
\item For $[a], [b]\in \mathbb{Z}_{12}$, define the group product as $[a][b] = [c]$, where $c= (a+b) \mod 12$.
\end{enumerate}
It is easy to verify that $\mathbb{Z}_{12}$ is a commutative group, and 
\begin{enumerate}
\item $\abs{[2]} = 6$ and $\abs{[4]} = 3$. Thus $\lcm(\abs{[2]}, \abs{[4]}) = 6$
\item $[2][4]=[6]$ and $\abs{[6]} = 2$.
\end{enumerate}
It shows that $\abs{[2][4]} \neq \lcm(\abs{[2]}, \abs{[4]})$.
\end{solution}


% Problem 1.14
\begin{problem}[1.14]
$\vartriangleright$ As a counterpoint to Exercise 1.13, prove that if $g$ and $h$ commute and
$\gcd(\abs{g}, \abs{h})=1$, then $\abs{gh}=\abs{g}\abs{h}$.(Hint: Let $N=\abs{gh}$; then $g^N=(h^{-1})^{N}$.What
can you say about this element?) [\S 1.6, 1.15, IV.2.5]
\end{problem}

\begin{solution}
As suggested in hint, we calculate 
$$1 = (gh)^N = g^Nh^N\Longrightarrow g^N = (h^{-1})^N.$$
Therefore, $g^N$ and $(h^{-1})^{N}$ has the same order, denoted by $t$. By Proposition 1.13, we have
$$t = \abs{g^N} = \frac{\abs{g}}{\gcd(N, \abs{g})} = \frac{\abs{h}}{\gcd(N, \abs{h})} = \abs{(h^{-1})^N}.$$
It is obvious that $t$ is a common divisor of $\abs{g}$ and $\abs{h}$. Since $\gcd(\abs{g}, \abs{h})=1$, $t$ must be $1$.
This means $g^N = 1 = (h^{-1})^N$. Thus
$$\abs{g} \mid N, \abs{h} \mid N\Longrightarrow \abs{g}\abs{h}\mid N.$$
On the other hand, since $(gh)^{\abs{g}\abs{h}} =1$, $N\mid\abs{g}\abs{h}$.

This proves that $N = \abs{g}\abs{h}$
\end{solution}

%Problem 1.15
\begin{problem}[1.15]
$\neg$ Let $G$ be a commutative group, and let $g \in G$ be an element of maximal
finite order, that is, such that if $h \in G$ has finite order, then $\abs{h} \leq \abs{g}$. 
Prove that in fact if $h$ has finite order in $G$, then $\abs{h}$ divides $\abs{g}$. (Hint: 
Argue by contradiction. If $\abs{h}$ is finite but does not divide $\abs{g}$, 
then there is a prime integer $p$ such that $\abs{g}=p^mr$, $\abs{h}=p^ns$, with $r$ and $s$ relatively 
prime to $p$ and $m < n$. Use Exercise 1.14 to compute the order of $g^{p^m}h^s$.) [\S 2.1, 4.11, IV.6.15]
\end{problem}

\begin{solution}
Applying prime factorization theorem to $\abs{g}$ and $\abs{h}$, we have
$$\abs{g} = p_1^{m_1}p_2^{m_2}\cdots p_l^{m_l}, \indent \abs{h} = p_1^{n_1}p_2^{n_2}\cdots p_l^{n_l},$$
where $m_i \geq 0, n_i \geq 0, i = 1, 2,\cdots l$.

If $\abs{h}\nmid\abs{g}$, then there is $p_j$ such that $n_j > m_j$. So, denoting $p_j, m_j, n_j$ by $p,m,n$ repsectively, 
we can rewrite 
$\abs{g}$ and $\abs{h}$ as
$$\abs{g} = p^{m}r, \abs{h} = p^ns$$
where $m < n$, $\gcd(r,p)=1, \gcd(s,p) = 1$.

By applying proposition 1.13, we have
$$\abs{g^{p^m}} = \frac{\abs{g}}{\gcd(p^m, \abs{g})} = r, \abs{h^s} = \frac{\abs{h}}{\gcd(s,\abs{h})} = p^n.$$
Since $\gcd(r, p^n) = 1$, the order of $g^{p^m}h^s$ must be $\abs{g^{p^m}}\abs{h^s} = p^nr$.i.e.
$$\abs{g^{p^m}h^s} = p^nr > p^mr=\abs{g}.$$
This contradicts the assumption that $g$ is the maximal order element in $G$. Hence $\abs{h}\mid\abs{g}$.
\end{solution}

\subsection{\textsection2. Examples of groups}

%Problem 2.1
\begin{problem}[2.1]
$\neg$One can associate an $n\times n$ matrix $M_\sigma$ with a permutation $\sigma \in S_n$, by
        letting the entry at $(i, \sigma(i))$ be 1, and letting all other entries be 0. For example,
        the matrix corresponding to the permutation
        \[
        \sigma=\left(
        \begin{matrix}  
        1 & 2 & 3\\     
    3 & 1 & 2   
        \end{matrix}
        \right)\in S_3
        \]
        would be 
        \[
        M_\sigma=\left(
        \begin{matrix}  
        0 & 0 & 1\\     
        1 & 0 & 0\\     
        0 & 1 & 0
        \end{matrix}
        \right)
        \]
        Prove that, with this notation,
        \[
        M_{\sigma\tau}=M_{\sigma}M_{\tau}
        \]
        for all $\sigma,\tau\in S_n$, where the product on the right is the ordinary product of matrices.
\end{problem}
\begin{solution}
$\sigma(i) = j$ $\iff$ the entry at $(i,j)$ position of $M_{\sigma}$ is $1$, the other entries on $i$-th row, $j$-th column are zero.\\
$\tau(j) = k$ $\iff$ the entry at $(j,k)$ position of $M_{\tau}$ is $1$, the other entries on $j$-th row, $k$-th column are zero.\\
The entry $(i,k)$ of $M_{\sigma}M_{\tau}$ is 1, the other entries in $i$-th row and $k$-th column are zero. This is the same as $M_{\sigma\tau}$ since 
$(\sigma\tau)(i) = \tau(\sigma(i)) = k$.\\

\noindent Applying this  argument with $i=1,2, \cdots, n$, we can conclude that $M_{\sigma\tau}=M_{\sigma}M_{\tau}$.
\end{solution}

%Problem 2.2
\begin{problem}[2.2]
 $\vartriangleright$ Prove that if $d \le n$, then $S_n$ contains elements of order $d$.[\S 2.1]
\end{problem}
\begin{solution}
The element 
 \[\sigma=\left(
        \begin{matrix}  
        1 & 2 & 3 & \cdots & d & (d+1) &\cdots & n\\     
    	d & 1 & 2 & \cdots & (d-1) &(d+1) &\cdots & n  
        \end{matrix}
        \right)
 \]
 has order $d$.  
\end{solution}

%Problem 2.3
\begin{problem}[2.3]
        For every positive integer $n$ find an element of order $n$ in $S_\mathbb{N}$.
\end{problem}
\begin{solution}
The element in $S_\mathbb{N}$ is an automorphism of $\mathbb{N}$. Define the automorphism $f:\mathbb{N}\to \mathbb{N}$ as the following:
\[
f(n)=
\begin{cases}
d-1, &n=0,\\
n-1, &1\leq n \leq d-1,\\
n, &n \geq d.
\end{cases}
\]
It is easy to verify that the order of $f$ is $d$.
\end{solution}


%Problem 2.4
\begin{problem}[2.4]
Define a homomorphism $D_8\to S_4$ by labelling vertices of a square, as we did for a triangle in \S 2.2. List the 8 permutation in the image
of this homomorphis.
\end{problem}
\begin{solution}
Label the vertices of square clockwise by $1,2,3,4$, then the counterclockwise rotations will be 
$$\sigma_1 = \left(
        \begin{matrix}  
        1 & 2 & 3 & 4\\     
    	1 & 2 & 3 & 4  
        \end{matrix}
        \right), \sigma_2 = \left(
        \begin{matrix}  
        1 & 2 & 3 & 4\\     
    	2 & 3 & 4 & 1  
        \end{matrix}
        \right), \sigma_3 = \left(
        \begin{matrix}  
        1 & 2 & 3 & 4\\     
    	3 & 4 & 1 & 2  
        \end{matrix}
        \right), \sigma_4 = \left(
        \begin{matrix}  
        1 & 2 & 3 & 4\\     
    	4 & 1 & 2 & 3  
        \end{matrix}
        \right).$$
        
        The reflections along the symmetric lines will be
        $$\sigma_5 = \left(
        \begin{matrix}  
        1 & 2 & 3 & 4\\     
    	1 & 4 & 3 & 2  
        \end{matrix}
        \right), \sigma_6 = \left(
        \begin{matrix}  
        1 & 2 & 3 & 4\\     
    	4 & 3 & 2 & 1  
        \end{matrix}
        \right), \sigma_7 = \left(
        \begin{matrix}  
        1 & 2 & 3 & 4\\     
    	3 & 2 & 1 & 4  
        \end{matrix}
        \right), \sigma_8 = \left(
        \begin{matrix}  
        1 & 2 & 3 & 4\\     
    	2 & 1 & 4 & 3  
        \end{matrix}
        \right).$$
 These are the images of $D_8$ in $S_4$ under the homomorphism.          
\end{solution}

%Problem 2.5
\hypertarget{Exercise II.2.5}{}
\begin{problem}[2.5]
 $\vartriangleright$ Describe generators and relations for all dihedral groups $D_{2n}$(Hint: Let $x$ be the reflection about a line through the 
 center of a regular $n$-gon and a vertex, and let $y$ be the counterclockwise rotation by $2\pi/n$. The group $D_{2n}$ will
 be generated by $x$ and $y$, subject to three relations. To see that these relations really determine $D_{2n}$, use them to show that any product
 $x^{i_1}y^{i_2}x^{i_3}h^{i_4}\cdots$ equals $x^iy^j$ for some $i,j$ with $0\leq i \leq 1, 0\leq j < n$)[8.4, \S IV 2.5]
\end{problem}
\begin{solution}
Let $x$ be the reflection about a line through the center of a regular $n$-gon and a vertex and $y$ be the counterclockwise rotation by $2\pi/n$, then we have the following rotations in $D_{2n}$: $$y, y^1, y^2, \cdots y^{n}.$$
It is obvious that there are no equal pairs, i.e. if $1\leq i,j \leq n$ and $i\neq j$, then $$y^i\neq y^j.$$
Otherwise, $y^{\abs{i-j}}$ be the identity, but this is impossible since vertices are rotated by angle $2\abs{i-j}\pi/n$ which is strictly less than $2\pi$. 

\noindent Now let's consider another set of elements:
$$xy, xy^2,\cdots xy^{n}.$$
We claim:
\begin{itemize}
\item For $0\leq i,j \leq n$ and $i\neq j$, then $xy^{i}\neq xy^j$. If they are equal, then $y^i=y^j$. It is impossible.
\item For $0\leq i,j \leq n$, $xy^i\neq y^j$. If $xy^i = y^j$, then $$x=y^k, \text{ where } k = \big(\max\{j-i, j+n-i\}\big)\mod{n}.$$ It is impossible since $x$ will map at least one vertex to itself while $y^k$ will change the position of all vertices.
\end{itemize}

Hence,
$D_{2n} =\{y, y^1,\cdots, y^n, xy, xy^2,\cdots, xy^n\}.$
\end{solution}

%Problem 2.6
\begin{problem}[2.6]
 $\vartriangleright$ For every positive integer $n$ construct a group containing elements $g, h$ such
that $\abs{g} = 2$, $\abs{h} = 2$, and $\abs{gh} = n$. (Hint: For $n > 1$, $D_{2n}$ will do.) [\S 1.6]
\end{problem}
\begin{solution}
Let's label the vertices of $n$-gon clockwise by $1,2,\cdots, n$. 
Let $g$ be the reflection about the line through the center and vertex $1$. It is easy to proof that $\abs{g} = 2$.
Let $p$ be a point defined as:
$$p=
\begin{cases}
\text{vertex } \frac{n+1}{2}, &\text{ if n is odd},\\
\text {middle point of edge from vertex }\frac{n}{2} \text { to } \frac{n}{2} + 1, &\text{ if n is even}. 
\end{cases}
$$
Let $f$ be the reflection about the line through center and $p$. It is also easy to verify $\abs{f} = 2$, but 
$$gf = \text{the counterclockwise rotation by }\frac{2\pi}{n}.$$
So $\abs{gf} = n$. 
\end{solution}

%Problem 2.7
\begin{problem}[2.7]
 $\neg$ Find all elements of $D_{2n}$ that commute with every other element. (The
parity of n plays a role.) [\S IV.1.2]
\end{problem}
\begin{solution}
Using the result of \hyperlink{Exercise II.2.5}{Exercise 2.5}, the dihedral group $D_{2n}$ can be written as
$$D_{2n}=\{e, y, y^2, \cdots, y^{n-1}, x, xy,\cdots, xy^{n-1}\}$$
Since $xy$ is a reflection, then $xyxy=e$, i.e. $xyx=y^{-1}$.

Let's discuss the following cases:\\
\underline{Rotation $y^k, 0\leq k\leq n-1$}:
\begin{itemize}
\item It is obvious that it commute with other rotations.
\item For reflection $xy^l$, if $xy^{l+k} = y^kxy^l$, then $y^{2k} = e$. So $2k\equiv 0\mod n$. If $n$ is odd, then $k=0$ and if $n$ is even then $k=\frac{n}{2}$. 
\end{itemize}
\underline{Reflection $xy^{k}, 0\leq k\leq n-1$}:
\begin{itemize}
\item For rotation $y^{l}$, if $xy{k+l} = y^{l}xy^k$, then $y^{2l} = e$. i.e. The reflection can only commute with some rotations, but not all.
\item For reflection $xy^l$, if $xy^kxy^l = xy^lxy^k$, then $y^{2(k+l)} = e$. So the reflection can only commute with some reflection, but not all. 
\end{itemize}

Hence, the center of $D_{2n}$ is $\{e\}$ if $n$ is odd, otherwise $\{e, y^{\frac{n}{2}}\}$.
\end{solution}

%Problem 2.8
\begin{problem}[2.8]
Find the orders of the groups of symmetries of the five ‘platonic solids’.
\end{problem}

%Problem 2.9
\begin{problem}[2.9]
Verify carefully that ‘congruence mod n’ is an equivalence relation.
\end{problem}
\begin{solution}
The proof is very straightforward:

\noindent \underline{\text{Reflexivity}}: Since $a-a=0=0n$, $a\equiv a \mod n$.

\noindent \underline{\text{Symmetry}}: If $a\equiv b \mod n$, then $\exists k\in\mathbb {Z}$ s.t. $a-b=kn$. Thus $b-a=(-k)n$. It implies $b\equiv a \mod n$.

\noindent \underline{\text{Transitivity}}: If $a\equiv b \mod n$, $b\equiv c \mod n$, then $\exists k_1, k_2\in \mathbb{Z}$ s.t. $a-b=k_1 n$, $b-c = k_2 n$. Therefore
$a-c = (k_1+k_2)n$. This proves $a\equiv c\mod n$.
\end{solution}


%Problem 2.10
\begin{problem}[2.10]
Prove that $\mathbb{Z}/n\mathbb{Z}$ consists of precisely $n$ elements.
\end{problem}

\begin{solution}
On one hand, let's consider set $$S:=\big{\{}[i]_n| 0\leq i\leq n-1\big{\}}.$$ $S$ contains $n$ distinct elements and $S\subset \mathbb{Z}/n\mathbb{Z}$. Therefore $\abs{\mathbb{Z}/n\mathbb{Z}}\geq n$.

On the other hand, for $m\in\mathbb{Z}$, we can rewrite it as $$m = \floor*{\frac{m}{n}} n + r, \text{ where } \floor*{\frac{m}{n}}\leq \frac{m}{n} < \floor*{\frac{m}{n}} + 1, 0\leq r\leq n-1.$$ Therefore $m\equiv r \mod n$. It implies $\abs{\mathbb{Z}/n\mathbb{Z}}\leq n$.\\
This proves $\abs{\mathbb{Z}/n\mathbb{Z}}=n$.
\end{solution}


%Problem 2.11
\begin{problem}[2.11]
 $\vartriangleright$ Prove that the square of every odd integer is congruent to $1$ modulo $8$.
[\S VII.5.1]
\end{problem}
\begin{solution}
Let $n = 2k+1$, then $$[n^2]_8=[4k(k+1) +1]_8 = [1]_8.$$ The last equation holds due to $2|k(k+1)$.
\end{solution}


%Problem 2.12
\begin{problem}[2.12]
Prove that there are no nonzero integers $a$, $b$, $c$ such that $a^2+b^2 = 3c^2$. (Hint:
By studying the equation $[a]^2_4 + [b]^2_4 = 3[c]^2_4$
in $\mathbb{Z}/4\mathbb{Z}$, show that $a$, $b$, $c$ would all have to be even. 
Letting $a = 2k$,$b = 2\ell$,$c=2m$, you would have $k^2 +\ell^2=3m^2$.
What's wrong with that?)
\end{problem}
\begin{solution}
Notice $[0]^2_4 =[2]^2_4=[0]_4, [1]^2_4=[3]^2_4=[1]_4$, if $a^2+b^2 = 3c^2$ has solution, then $[a]_4^2+[b]^2_4 = 3[c]_4^2$ has the same solution.
So both hand of this equation will be $[0]_4$, i.e. $a,b,c$ are even numbers. Letting $a = 2k$,$b = 2\ell$,$c=2m$, we have
$$4k^2+4\ell^2=4(3m^2),$$ thus $$k^2 +\ell^2=3m^2.$$ 

Now we can assume that $k, \ell, m$ are odd numbers or zeros
otherwise repeat the procedure above to cancel the factor $2$. Therefore, $[k]^2_4+[\ell]^2_4$ could be $[0]_4, [1]_4, [2]_4$ while $3[m]^2_4$ could be $[0]_4,[3]_4$.
The equation holds only if they are $[0]_4$. i.e. $k=\ell=m=0$. This proves only zeros satisfy the equation $a^2+b^2 = 3c^2$.
\end{solution}


%Problem 2.13
\hypertarget{Exercise II.2.13}{}
\begin{problem}[2.13]
$\vartriangleright$ Prove that if $\gcd(m, n) = 1$, then there exist integers $a$ and $b$ such that
$$am + bn = 1.$$ 
(Use Corollary 2.5.) Conversely, prove that if $am+bn = 1$ for some integers $a$ and $b$,
then $\gcd(m, n) = 1$. [2.15, \S V.2.1, V.2.4]
\end{problem}

\begin{solution}
If $\gcd(m, n) = 1$, then $[m]_n$ generates $\mathbb{Z}/n\mathbb{Z}$, i.e.
$$\mathbb{Z}/n\mathbb{Z} = \{[m]_n, 2[m]_n, \cdots, (n-1)[m]_n\}.$$
Thus, $\exists k\in \mathbb{N}$ such that $$k[m]_n = [1]_n.$$ This is equivalent to $\exists a,b\in\mathbb{Z}$ such that 
$$am + bn = 1.$$

Conversely, It is obvious that $1 | \gcd(m,n)$. On the other hand, if $am+bn = 1$, then $\gcd(m,n) | 1$. It proves that $\gcd(m,n)=1$.
\end{solution}

%Problem 2.14
\begin{problem}[2.14]
$\vartriangleright$ State and prove an analog of Lemma 2.2, showing that the multiplication
on $\mathbb{Z}/n\mathbb{Z}$ is a well-defined operation. [\S 2.3, \S III.1.2]
\end{problem}
\begin{solution}
\underline{Statement} If $a\equiv a' \mod n$ and $b\equiv b'\mod n$, then $ab \equiv a'b'\mod n$.\\
\underline{Proof}: since $a\equiv a' \mod n$ and $b\equiv b'\mod n$, there exist $k,l\in\mathbb{Z}$ such that
$$a=a'+kn, b=b'+ln.$$
Thus 
$ab=(a'+kn)(b'+ln) = a'b'+(a'l+kb'+kln)n$. It implies $$[a]_n[b]_n=[ab]_n=[a'b']_n=[a']_n[b']_n.$$
i.e. the multiplication on $\mathbb{Z}/n\mathbb{Z}$ is well-defined.
\end{solution}


%Problem 2.15
\begin{problem}[2.15]
$\neg$ Let $n > 0$ be an odd integer.
  \begin{itemize}
    \item Prove that if $\gcd(m, n) = 1$, then $\gcd(2m + n, 2n) = 1$. (Use \hyperlink{Exercise II.2.13}Exercise 2.13.)
    \item Prove that if $\gcd(r, 2n) = 1$, then $\gcd(\frac{r+n}{2}, n) = 1$. (Ditto.)
    \item Conclude that the function $[m]_n\rightarrow[2m + n]_{2n}$ is a bijection between $(\mathbb{Z}/n\mathbb{Z})^*$ and $(\mathbb{Z}/2n\mathbb{Z})^*$.
  \end{itemize}
The number $\phi(n)$ of elements of $(\mathbb{Z}/n\mathbb{Z})^*$ is Euler’s $\phi(n)$-function. The reader has just proved that if $n$ is odd, then $\phi(2n) = \phi(n)$. Much more general formulas will be given later on (cf. \hyperlink{Exercise V.6.8}{Exercise V.6.8}). [VII.5.11]
\end{problem}
\begin{solution}
  \begin{itemize}
    \item Suppose $d=\gcd(2m+n,2n)$. Since $2m+n$ is an odd number and $d \mid (2m+n)$, $d \nmid 2$. Thus $d\mid n$,  $d | m$. It implies $d \mid \gcd(m,n)$. Hence $d=1$.
    \item If $\gcd(r, 2n) = 1$, then there exist $a,b\in\mathbb{Z}$ such that 
    $$ar+b(2n) = 1\Rightarrow 2a\big{(}\frac{r+n}{2}\big{)} + (2b-a)n = 1\Rightarrow \gcd\big{(}\frac{r+n}{2}, n\big{)} = 1.$$
    \item Let's consider $n > 1$ and $n$ is odd. The function $f: (\mathbb{Z}/n\mathbb{Z})^* \to (\mathbb{Z}/2n\mathbb{Z})^*$ defined as 
    $$f([m]_n) = [2m + n]_{2n}$$ is well-defined. This is because if $[m_1]_n = [m_2]_n$, then 
    $$2(m_1-m_2) = 2kn \Rightarrow (2m_1+n) - (2m_2+n) = 2kn \Rightarrow [2m_1+n]_{2n} = [2m_2+n]_{2n}.$$ 
    By the previous results, $f$ is a bijection.
   \end{itemize} 
   It proves all results.
\end{solution}

%Problem 2.16
\begin{problem}[2.16]
Find the last digit of $1238237^{18238456}$. (Work in $\mathbb{Z}/10\mathbb{Z}$.)
\end{problem}
\begin{solution}
$$[1238237^{18238456}]_{10} = [7^{18238456}]_{10} = [1^{4559614}]_{10}=[1]_{10}.$$
Thus the last digit of $1238237^{18238456}$ is $1$. 
\end{solution}

%Problem 2.17
\begin{problem}[2.17]
$\vartriangleright$ Show that if $m\equiv m'\mod n$, then $\gcd(m,n) = 1$ if and only if $\gcd(m',n)=1$. [\S 2.3]
\end{problem}
\begin{solution}
If $m \equiv m'\mod n$,then there is $k\in\mathbb{Z}$ such that $m=m'+kn$.
Thus
$$\gcd(m, n) = 1 \iff \exists a, b\in\mathbb{Z}\text{ s.t. } am+bn=1\iff am^{\prime}+(b+k)n =1\iff \gcd(m^{\prime},n)=1.$$
This proves the statement.
\end{solution}

%Problem 2.18
\begin{problem}[2.18]
For $d \leq n$, define an injective function $\mathbb{Z}/d\mathbb{Z} \to S_n$ preserving the operation,
that is, such that the sum of equivalence classes in $\mathbb{Z}/d\mathbb{Z} \to S_n$ corresponds to the product
of the corresponding permutations.
\end{problem}
\begin{solution}
Let \[\sigma=\left(
        \begin{matrix}  
        1 & 2 & 3 & \cdots & d & (d+1) &\cdots & n\\     
    	d & 1 & 2 & \cdots & (d-1) &(d+1) &\cdots & n  
        \end{matrix}
        \right)
 \]
Let $f: \mathbb{Z}/d\mathbb{Z} \to S_n$ defined as 
$$f([k]_d) = \sigma^d.$$
This map will preserve the operations.
\end{solution}


%Problem 2.19
\begin{problem}[2.19]
$\vartriangleright$ Both $(\mathbb{Z}/5\mathbb{Z})^{\ast}$ and $(\mathbb{Z}/12\mathbb{Z})^{\ast}$ consist of $4$ elements.
Write their multiplication tables, and prove that no re-ordering of the elements will make them match.
(Cf. Exercise 1.6.) [\S 4.3]
\end{problem}
\begin{solution}
$$(\mathbb{Z}/5\mathbb{Z})^{\ast} = \{[1]_5,[2]_5,[3]_5, [4]_5\}.$$ Each non-unit element in $(\mathbb{Z}/5\mathbb{Z})^{\ast}$ could generate the entire group.
$$(\mathbb{Z}/12\mathbb{Z})^{\ast} =\{[1]_{12}, [5]_{12}, [7]_{12}, [11]_{12}\}.$$ Each non-unit element in $(\mathbb{Z}/5\mathbb{Z})^{\ast}$ has order $2$. 
There is no isomorphism between these two groups.
\end{solution}


\subsection{\textsection3. The category $\mathsf{Grp}$}

%Problem 3.1
\begin{problem}[3.1]
$\vartriangleright$ Let $\phi: G \to H$ be a morphism in a category $\mathsf{C}$ with products. Explain why
there is a unique morphism $(\phi \times \phi): G \times G \to H \times H$ compatible in the evident
way with the natural projections.
(This morphism is defined explicitly for $\mathsf{C} = \mathsf{Set}$ in \S 3.1.) [\S 3.1, 3.2]
\end{problem}
\begin{solution}
\end{solution}

%Problem 3.2
\begin{problem}[3.2]
  Let $\varphi : G\rightarrow H, \psi : H \rightarrow K$ be morphisms in a category with products, and
  consider morphisms between the products $G\times G, H\times H, K\times K$ as in Exercise 3.1.
  Prove that
  \[
  (\psi\varphi) \times(\psi\varphi)=(\psi \times \psi)(\varphi\times \varphi) .
  \]
  (This is part of the commutativity of the diagram displayed in \textsection 3.2.)
\end{problem}
\begin{solution}
\end{solution}

%Problem 3.3
\hypertarget{Exercise II.3.3}{}
\begin{problem}[3.3]
  Show that if $G, H$ are abelian groups, then $G \times H$ satisfies the universal property for coproducts in $\mathsf{Ab}$.
\end{problem}
\begin{solution}
Let's define the operation $\ast$ on $G\times H$ as the following:
$$(g_1,h_1)\ast(g_2,h_2) = (g_1g_2, h_1h_2).$$ It is easy to verify that $(G\times H, \ast)$ is an abelian group.

\noindent Define the inclusion functions
$$\iota_G: G \to G\times H, \iota_G(g)=(g, e_H),$$
$$\iota_H: H \to G\times H, \iota_G(h)=(e_G, h).$$
It is not complicate to verify that both $\iota_G$ and $\iota_H$ are group homomorphisms.

\noindent For homomorphisms $\phi_G: G\to A$ and $\phi_H: H\to A$, we can define a set-function 
$$\phi_{GH}: G\times H \to A, \phi_{GH}((g,h))=\phi_G(g)\phi_H(h).$$
For $g\in G$, $$(\iota_G\phi_{GH})(g) = \phi_{GH}(\iota_G(g)) = \phi_{GH}((g, e_H))=\phi_G(g)\phi_H(e_H)=\phi_G(g).$$
For $h\in H$, $$(\iota_H\phi_{GH})(h) = \phi_{GH}(\iota_H(h)) = \phi_{GH}((e_G, h))=\phi_G(e_G)\phi_H(h)=\phi_H(h).$$

\noindent Now, we prove $\phi_{GH}$ is a group homomorphism. In fact, $\forall (g_1,h_1), (g_2,h_2)\in G\times H$
\begin{equation*}\begin{split}
\phi_{GH}((g_1, h_1)\ast(g_2,h_2)) &= \phi_{GH}((g_1g_2,h_1h_2))\\&=\phi_G(g_1g_2)\phi_H(h_1h_2) \\&= (\phi_G(g_1)\phi_H(h_1))(\phi_G(g_2)\phi_H(h_2)) \\&=\phi_{GH}((g_1,h_1))\phi_{GH}((g_2,h_2))
\end{split}
\end{equation*}
So, we have proved that $G\times H$ is the coproduct in $\mathsf{Ab}$. 
\end{solution}

%Problem 3.4
\begin{problem}[3.4]
  Let $G, H$ be groups, and assume that $G\cong H\times G$. Can you conclude that $H$ is trivial? (Hint: No. Can you construct a counterexample?)
\end{problem}
\begin{solution}
\end{solution}

%Problem 3.9
\begin{problem}[3.9]
Show that fiber products and coproducts exist in $\mathsf{Ab}$. (Cf. Exercise I.5.12. For
coproducts, you may have to wait until you know about $\mathit{quotients}$.)
\end{problem}
\begin{solution}
\underline{Fiber product}: For $A, B, C$ in $\mathsf{Ab}$, let's define the following product structure:
\[
\begin{tikzcd}
& A \arrow[dr, "\alpha"]& \\
A\times_C B \arrow[ur, "\pi_A"] \arrow[dr, "\pi_B"'] & & Z\\
& B \arrow[ur, "\beta"'] &
\end{tikzcd}
\]
where $$A\times_C B=\{(a,b)| \alpha(a)=\beta(b)\}\subset A\times B$$ and $$\pi_A: A\times_C B \to A, \pi_A((a,b)) = a,$$ 
$$\pi_B: A\times_C B \to B, \pi_B((a,b)) = b.$$ 
It is not hard to prove:
\begin{itemize}
\item For $\forall (a_1,b_1)\in A\times_C B, (a_2,b_2)\in A\times_C B$, define operation $\ast$ as  $$(a_1,b_1)\ast (a_2,b_2) = (a_1a_2, b_1b_2).$$
Equipped with $\ast$, $A\times_C B$ is an abelian group.
\item It is obvious that $\pi_A: A\times_C B \to A$ and $\pi_B: A\times_C B \to B$ are group homomorphisms.
\item By definition, $\forall (a,b)\in A\times_C B$, $$(\alpha\pi_A)((a,b)) = \alpha(a) = \beta(b) = (\beta\pi_B)((a,b)).$$
\end{itemize}

$\forall Z$ in $\mathsf{Ab}$ and the group homomorphisms $f_A: Z\to A$, f$_B: Z\to B$. If $\alpha f_A = \beta f_B$, then we can define $\sigma: Z\to A\times_C B$ such that the following diagram commutative:
\[
\begin{tikzcd}
& & A \arrow[dr, "\alpha"]& \\
Z\arrow[r, "\sigma"] \arrow[urr, bend left, "f_A" near end] \arrow[drr, bend right, "f_B"' near end] &A\times_C B \arrow[ur, "\pi_A"] \arrow[dr, "\pi_B"'] & & Z\\
&& B \arrow[ur, "\beta"'] &
\end{tikzcd}
\]
where $\sigma(z) = (f_A(z), f_B(z))$. It is not hard to prove that $\sigma$ is a group homomorphism and $f_A = \pi_A\sigma, f_B=\pi_B\sigma$. It proves that $A\times_C B$ with the standard projections is the fibered product of the category $\mathsf{Ab}$.

\noindent \underline{Fiber coproduct}: For $C, A, B$ in $\mathsf{Ab}$ and the homomorphisms $\alpha: C\to A, \beta: C\to B$, let's consider the subgroup 
$\ker(\alpha)\cap \ker(\beta)\subset C$ and the quotient group $C/\big{(}\ker(\alpha)\cap \ker(\beta)\big{)}$.
 
 
 
let's define the following structure: 
$$A\amalg_C B=
\begin{cases}
(0, a), &\text{ if } a\notin \alpha(C)\subset A \\
(1, b), &\text{ if } b \notin \beta(C)\subset B \\
(0,a)\sim (1,b), &\text{ if } a=\alpha(c), b=\beta(c).
\end{cases}$$
i.e. in $A\amalg_C B$,  $(0,a)=(1,b)$, if there is $c$ such that $a=\alpha(c), b=\beta(c)$.
\begin{itemize}
\item Define an operation $\ast$ on $A\amalg_C B$ as the following:

\end{itemize}

\end{solution}










