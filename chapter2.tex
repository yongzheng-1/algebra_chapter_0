\section{Chapter II.\hspace{0.2em} Groups, first encounter}

\subsection{\textsection1. Definition of group}

%Problem 1.1
\begin{problem}[1.1]
$\vartriangleright$ Write a careful proof that every group is the group of isomorphisms of a
groupoid. In particular, every group is the group of automorphisms of some object
in some category.[\S 2.1]
\end{problem}

\begin{solution}

\end{solution}

%Problem 1.2
\begin{problem}[1.2]
$\vartriangleright$ Consider the `sets of numbers' listed in \S 1.1, and decide which are made into
groups by conventional operations such as $+$ and $\cdot$. Even if the answer is negative
(for example, $(R, \cdot)$ is not a group), see if variations on the definition of these sets
lead to groups (for example, $(R^{\ast}, \cdot)$ is a group; cf. \S 1.4). [\S 1.2]
\end{problem}

\begin{solution}
\begin{enumerate}
\item $\mathbb{N}$ is not a group under operation $+$ or $\cdot$.
\item $\mathbb{Z}$ is a group under operation $+$, but not under operation $\cdot$.
\item $\mathbb{N}^{\ast}$ and $\mathbb{Z}^{\ast}$ can't be a group under operation $\cdot$.
\item $\mathbb{Q}, \mathbb{R}$ and $\mathbb{C}$ are groups under operation $+$, but can't be groups under operation $\cdot$.
\item $\mathbb{Q}^{\ast}, \mathbb{R}^{\ast}$ and $\mathbb{C}^{\ast}$ are groups under operation $\cdot$.
\end{enumerate}
These statements are easy to be verified. 
\end{solution}

% Problem 1.3
\begin{problem}[1.3]
Prove that $(gh)^{-1}=h^{-1}g^{-1}$ for all elements $g,h$ of a group $G$.
\end{problem}

\begin{solution}
Since $$(gh)(h^{-1}g^{-1}) = g(hh^{-1})g^{-1} = gg^{-1} = e_G$$ and $$(h^{-1}g^{-1})(gh) = h^{-1}(g^{-1}g)h=h^{-1}h=e_G,$$
it turns out $(gh)^{-1}=h^{-1}g^{-1}$.
\end{solution}

% Problem 1.4
\begin{problem}[1.4]
Suppose that $g^2 = e$ for all elements $g$ of a group $G$; prove that $G$ is commutative.
\end{problem}

\begin{solution}
By assumption, for all elements $g,h\in G$
$$hghg=e=gg$$
it means $hgh=g$ then $hghh=gh$. It implies $hg=gh$.i.e. $G$ is commutative.
\end{solution}

% Problem 1.5
\begin{problem}[1.5]
  The `multiplication table' of a group is an array compiling the results of all
  multiplications $g \bullet h$:
\begin{center}
\begin{tabular}{c||c|c|c|c}

  $\bullet$ & $e$ & $\cdots$ & $h$ & $\cdots$ \\ \hline \hline
  $e$ & $e$ & $\cdots$ & $h$ & $\cdots$ \\ \hline
  $\cdots$ & $\cdots$ & $\cdots$ & $\cdots$ & $\cdots$ \\ \hline
  $g$ & $g$ & $\cdots$ & $g\bullet h$ & $\cdots$ \\ \hline
  $\cdots$ & $\cdots$ & $\cdots$ & $\cdots$ & $\cdots$ \\
\end{tabular}
\end{center}
  (Here $e$ is the identity element. Of course the table depends on the order in
  which the elements are listed in the top row and leftmost column.) Prove that
  every row and every column of the multiplication table of a group contains all
  elements of the group exactly once (like Sudoku diagrams!).
\end{problem}

\begin{solution}
If $f\bullet g = f \bullet h$, then $g=h$. It is a contradiction.Hence there is no equal elements in $f$-row. 
If $g\bullet f= h \bullet f$, then $g=h$. It is a contradiction too. Hence there is no equal elements in $f$-column. 
\end{solution}

%Problem 1.6
\begin{problem}[1.6]
$\neg$ Prove that there is only one possible multiplication table for $G$ if $G$ has
exactly $1$, $2$, or $3$ elements. Analyze the possible multiplication tables for groups
with exactly $4$ elements, and show that there are two distinct tables, up to reordering
the elements of $G$. Use these tables to prove that all groups with $\leq 4$ elements are commutative.

(You are welcome to analyze groups with $5$ elements using the same technique,
but you will soon know enough about groups to be able to avoid such brute-force
approaches.) [2.19]
\end{problem}

\begin{solution}
If $\abs{G}=1$, then $G=\{1\}$. It is trivial group.

\noindent If $G=\{1, a\}$, then $a^2=1$. 

\noindent If $G=\{1, a, b\}$, then $ab=1$. It implies $a^2=b$. Hence $G$ is a cyclic group $\{1, a, a^2\}$. 

\noindent Now, let consider $G=\{1,a,b,c\}$
	
	If one of $a^2, b^2, c^2$ is not $1$, say, $a^2\neq 1$, then $a^3\neq 1$ and $G=\{1,a,a^2, a^3\}$ is a cyclic group. In fact, suppose $a^3=1$. 
	The possible values of $ab$ are $1$ or $c$. If $ab=1$, then $b=a^2$ and $ac=1$ because $ac=b$ implies $c=a$, this contradicts to our assumption. But if $ac=1$, $ab=1$, then $b=c$ also contradicts to our assumption. 
Therefore, we must have $ab=c, ac=b$. This leads to $a^2=1$ and contradicts to our assumption. Hence $a^3\neq 1$.

If $a^2=b^2=c^2=1$, then $G=\{1, a, b, ab\}$.
\end{solution}

% Problem 1.7
\begin{problem}[1.7]
Prove Corollary 1.11.
\end{problem}

\begin{solution}
($\implies$): This is Lemma 1.10.\\
($\impliedby$): Suppose $N=n\abs{g}$, then $g^N = (g^{\abs{g}})^n = e^n=e$.
\end{solution}

% Problem 1.8
\begin{problem}[1.8]
$\neg$ Let $G$ be a finite abelian group with exactly one element $f$ of order $2$. Prove
that $\prod_{g\in G}g=f$.[4.16]
\end{problem}

\begin{solution}
By assumption, there are only two elements $1, f$ such that $1^{-1} = 1, f^{-1} = f$. Hence 
$$\prod_{g\in G}g = (\prod_{g\in G\setminus \{1,f\}}g) \bullet f = (\underbrace{1\bullet 1\bullet \cdots 
\bullet 1}_{\frac{\abs{G} -2}{2}}) \bullet f = f.$$
The statement is proved.
\end{solution}

% Problem 1.9
\begin{problem}[1.9]
Let $G$ be a finite group, of order $n$, and let $m$ be the number of elements $g \in G$
of order exactly $2$. Prove that $n-m$ is odd. Deduce that if $n$ is even, then
$G$ necessarily contains elements of order $2$.
\end{problem}

\begin{solution}
Let's rewrite set $G$ as
$$G = \{1\} \cup \{g\big{|} g\in G, \abs{g} = 2\}\cup \{g\big{|} g\in G, \abs{g} > 2\}.$$ 
If $\abs{g} > 2$, then $\abs{g^{-1}} > 2$ and $g\neq g^{-1}$. Hence $\abs{\{g\big{|} g\in G, \abs{g} > 2\}} = 2k,k\in \mathbb{N}$ and 
$n=1+m+2k$. This is $n-m=2k+1$, an odd number. 

If $n$ is even number, then $m = n - (2k+1) > 0$, i.e. $G$ contain elements of order $2$. 
\end{solution}

% Problem 1.10
\begin{problem}[1.10]
Suppose the order of $g$ is odd. What can you say about the order of $g^2$?
\end{problem}

\begin{solution}
If $\abs{g}$ is odd, then $\gcd(2,\abs{g}) = 1$. Hence $\abs{g^2} = \frac{\abs{g}}{\gcd(2,\abs{g})} = \abs{g}$
\end{solution}

% Problem 1.11
\begin{problem}[1.11]
Prove that for all $g,h$ in a group $G$, $\abs{gh}=\abs{hg}$.(Hint: Prove that $\abs{aga^{-1}}=\abs{g}$ for all $a,g$ in G.)
\end{problem}

\begin{solution}
For $h,g\in G$, 
$$(hgh^{-1})^{\abs{g}} = hg^{\abs{g}}h^{-1} = 1.$$ Therefore $\abs{hgh^{-1}}\mid \abs{g}$.
On the other hand, $g=h^{-1}(hgh^{-1})h$ implies $\abs{g} \mid \abs{hgh^{-1}}$.
This proves $\abs{hgh^{-1}}=\abs{g}$. Taking $g=gh$, we have 
$\abs{hg} = \abs{hghh^{-1}}=\abs{gh}$
\end{solution}

% Problem 1.12
\begin{problem}[1.12]
$\vartriangleright$In the group of invertible $2\times 2$ matrices, consider
$$g=\left(\begin{array}{cc} 0 & -1 \\ 1 & 0 \end{array} \right), \indent h=\left(\begin{array}{cc} 0 & 1 \\ -1 & -1 \end{array} \right).$$
Verify that $\abs{g}=4$, $\abs{h}=3$ and $\abs{gh}=\infty$.[\S 1.6]
\end{problem}

\begin{solution}
$g \neq I, g^2=\left(\begin{array}{cc} -1 & 0 \\ 0 & -1 \end{array} \right)\neq I, 
g^3=\left(\begin{array}{cc} 0 & 1 \\ -1 & 0 \end{array} \right)\neq I, g^4 = I$.
So $\abs{g}=4$.\\
$h\neq I, h^2=\left(\begin{array}{cc} -1 & -1 \\ 1 & 0 \end{array} \right)\neq I, h^3 = I$, So $\abs{h} = 3$\\
$gh=\left(\begin{array}{cc} 1 & 1 \\ 0 & 1 \end{array} \right), (gh)^n = \left(\begin{array}{cc} 1 & n \\ 0 & 1 \end{array} \right)$
So. $\abs{gh} = \infty$.
\end{solution}

% Problem 1.13
\begin{problem}[1.13]
$\vartriangleright$Give an example showing that $\abs{gh}$ is not necessarily equal to $\lcm(\abs{g},\abs{h})$, even if $g$ and $h$
commute.[\S 1.6, 1.14]
\end{problem}

\begin{solution}
Let's consider the group $\mathbb{Z}_{12}$ defined as the following:
\begin{enumerate}
\item $\mathbb{Z}_{12} = \{[0], [1], [2],\cdots, [11]\}$
\item For $[a], [b]\in \mathbb{Z}_{12}$, define the group product as $[a][b] = [c]$, where $c= (a+b) \mod 12$.
\end{enumerate}
It is easy to verify that $\mathbb{Z}_{12}$ is a commutative group, and 
\begin{enumerate}
\item $\abs{[2]} = 6$ and $\abs{[4]} = 3$. Thus $\lcm(\abs{[2]}, \abs{[4]}) = 6$
\item $[2][4]=[6]$ and $\abs{[6]} = 2$.
\end{enumerate}
It shows that $\abs{[2][4]} \neq \lcm(\abs{[2]}, \abs{[4]})$.
\end{solution}


% Problem 1.14
\begin{problem}[1.14]
$\vartriangleright$ As a counterpoint to Exercise 1.13, prove that if $g$ and $h$ commute and
$\gcd(\abs{g}, \abs{h})=1$, then $\abs{gh}=\abs{g}\abs{h}$.(Hint: Let $N=\abs{gh}$; then $g^N=(h^{-1})^{N}$.What
can you say about this element?) [\S 1.6, 1.15, IV.2.5]
\end{problem}

\begin{solution}
As suggested in hint, we calculate 
$$1 = (gh)^N = g^Nh^N\Longrightarrow g^N = (h^{-1})^N.$$
Therefore, $g^N$ and $(h^{-1})^{N}$ has the same order, denoted by $t$. By Proposition 1.13, we have
$$t = \abs{g^N} = \frac{\abs{g}}{\gcd(N, \abs{g})} = \frac{\abs{h}}{\gcd(N, \abs{h})} = \abs{(h^{-1})^N}.$$
It is obvious that $t$ is a common divisor of $\abs{g}$ and $\abs{h}$. Since $\gcd(\abs{g}, \abs{h})=1$, $t$ must be $1$.
This means $g^N = 1 = (h^{-1})^N$. Thus
$$\abs{g} \mid N, \abs{h} \mid N\Longrightarrow \abs{g}\abs{h}\mid N.$$
On the other hand, since $(gh)^{\abs{g}\abs{h}} =1$, $N\mid\abs{g}\abs{h}$.

This proves that $N = \abs{g}\abs{h}$
\end{solution}

%Problem 1.15
\begin{problem}[1.15]
$\neg$ Let $G$ be a commutative group, and let $g \in G$ be an element of maximal
finite order, that is, such that if $h \in G$ has finite order, then $\abs{h} \leq \abs{g}$. 
Prove that in fact if $h$ has finite order in $G$, then $\abs{h}$ divides $\abs{g}$. (Hint: 
Argue by contradiction. If $\abs{h}$ is finite but does not divide $\abs{g}$, 
then there is a prime integer $p$ such that $\abs{g}=p^mr$, $\abs{h}=p^ns$, with $r$ and $s$ relatively 
prime to $p$ and $m < n$. Use Exercise 1.14 to compute the order of $g^{p^m}h^s$.) [\S 2.1, 4.11, IV.6.15]
\end{problem}

\begin{solution}
Applying prime factorization theorem to $\abs{g}$ and $\abs{h}$, we have
$$\abs{g} = p_1^{m_1}p_2^{m_2}\cdots p_l^{m_l}, \indent \abs{h} = p_1^{n_1}p_2^{n_2}\cdots p_l^{n_l},$$
where $m_i \geq 0, n_i \geq 0, i = 1, 2,\cdots l$.

If $\abs{h}\nmid\abs{g}$, then there is $p_j$ such that $n_j > m_j$. So, denoting $p_j, m_j, n_j$ by $p,m,n$ repsectively, 
we can rewrite 
$\abs{g}$ and $\abs{h}$ as
$$\abs{g} = p^{m}r, \abs{h} = p^ns$$
where $m < n$, $\gcd(r,p)=1, \gcd(s,p) = 1$.

By applying proposition 1.13, we have
$$\abs{g^{p^m}} = \frac{\abs{g}}{\gcd(p^m, \abs{g})} = r, \abs{h^s} = \frac{\abs{h}}{\gcd(s,\abs{h})} = p^n.$$
Since $\gcd(r, p^n) = 1$, the order of $g^{p^m}h^s$ must be $\abs{g^{p^m}}\abs{h^s} = p^nr$.i.e.
$$\abs{g^{p^m}h^s} = p^nr > p^mr=\abs{g}.$$
This contradicts the assumption that $g$ is the maximal order element in $G$. Hence $\abs{h}\mid\abs{g}$.
\end{solution}

\subsection{\textsection2. Examples of groups}

%Problem 2.1
\begin{problem}[2.1]
$\neg$One can associate an $n\times n$ matrix $M_\sigma$ with a permutation $\sigma \in S_n$, by
        letting the entry at $(i, \sigma(i))$ be 1, and letting all other entries be 0. For example,
        the matrix corresponding to the permutation
        \[
        \sigma=\left(
        \begin{matrix}  
        1 & 2 & 3\\     
    3 & 1 & 2   
        \end{matrix}
        \right)\in S_3
        \]
        would be 
        \[
        M_\sigma=\left(
        \begin{matrix}  
        0 & 0 & 1\\     
        1 & 0 & 0\\     
        0 & 1 & 0
        \end{matrix}
        \right)
        \]
        Prove that, with this notation,
        \[
        M_{\sigma\tau}=M_{\sigma}M_{\tau}
        \]
        for all $\sigma,\tau\in S_n$, where the product on the right is the ordinary product of matrices.
\end{problem}
\begin{solution}
$\sigma(i) = j$ $\iff$ the entry at $(i,j)$ position of $M_{\sigma}$ is $1$, the other entries on $i$-th row, $j$-th column are zero.\\
$\tau(j) = k$ $\iff$ the entry at $(j,k)$ position of $M_{\tau}$ is $1$, the other entries on $j$-th row, $k$-th column are zero.\\
The entry $(i,k)$ of $M_{\sigma}M_{\tau}$ is 1, the other entries in $i$-th row and $k$-th column are zero. This is the same as $M_{\sigma\tau}$ since 
$(\sigma\tau)(i) = \tau(\sigma(i)) = k$.\\

\noindent Applying this  argument with $i=1,2, \cdots, n$, we can conclude that $M_{\sigma\tau}=M_{\sigma}M_{\tau}$.
\end{solution}

%Problem 2.2
\begin{problem}[2.2]
 $\vartriangleright$ Prove that if $d \le n$, then $S_n$ contains elements of order $d$.[\S 2.1]
\end{problem}
\begin{solution}
The element 
 \[\sigma=\left(
        \begin{matrix}  
        1 & 2 & 3 & \cdots & d & (d+1) &\cdots & n\\     
    	d & 1 & 2 & \cdots & (d-1) &(d+1) &\cdots & n  
        \end{matrix}
        \right)
 \]
 has order $d$.  
\end{solution}

%Problem 2.2
\begin{problem}[2.3]
        For every positive integer $n$ find an element of order $n$ in $S_\mathbb{N}$.
\end{problem}
\begin{solution}
The element in $S_\mathbb{N}$ is an automorphism of $\mathbb{N}$. Define the automorphism $f:\mathbb{N}\to \mathbb{N}$ as the following:
\[
f(n)=
\begin{cases}
d-1, &n=0,\\
n-1, &1\leq n \leq d-1,\\
n, &n \geq d.
\end{cases}
\]
It is easy to verify that the order of $f$ is $d$.
\end{solution}
