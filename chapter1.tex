\section{Chapter I.\hspace{0.2em} Preliminaries: Set theory \\ and categories}

\subsection{\textsection1. Naive Set Theory}

% Problem 1.1
\begin{problem}[1.1]
	Locate a discussion of Russel's paradox, and understand it.
\end{problem}

\begin{solution}
Naive set theory assumes the so-called naive or unrestricted comprehension principle: For any formula 
$\phi(x)$ containing $x$ as a free variable, there will exist the set $\{x: \phi(x)\}$ whose members are
exactly those objects that satisfy $\phi(x)$. Let's assume $\phi(x)$ be $x\notin x$ and 
$R = \{x: x\notin x\}$. Is $R$ a member of itself?. If $R\notin R$, 
then the condition is satisfied. It implies $R\in R$. This is a contradiction. On the other hand, if $R\in R$,
then $R$ must satisfy the condition, i.e. $R\notin R$. This is also a contradiction.
\end{solution}

% Problem 1.2
\hypertarget{Exercise I.1.2}{}
\begin{problem}[1.2]
	$\vartriangleright$ Prove that if $\sim$ is an equivalence relation on a set $S$, then
	the corresponding family $\mathscr{P}_{\sim}$ defined in \S1.5 is indeed a
	partition of $S$; that is, its elements are nonempty, disjoint, and their union
	is $S$. [\S1.5]
\end{problem}

\begin{solution}
By the reflexivity, $\forall a\in S$, $a\in [a]_{\sim}$. So $$\bigcup_{[v]_{\sim}\in \mathscr{P}_{\sim}}[v]_{\sim} = S.$$
Suppose $c\in [a]_{\sim}\cap[b]_{\sim}$, then $a \sim c, c\sim b$. By transitivity, $a\sim b$.
$\forall x\in [b]_{\sim}$, $b\sim x$. By transitivity again, $a\sim x$. It implies $[b]_{\sim}\subset[a]_{\sim}$. Similarly,
$[a]_{\sim}\subset[b]_{\sim}$ holds too. Therefore, 
$$\text{if } [a]_{\sim} \cap [b]_{\sim}\neq \emptyset \Rightarrow [a]_{\sim} = [b]_{\sim}.$$
It proves that $\mathscr{P}_{\sim}$ is a partition of $S$.
\end{solution}

% Problem 1.3
\hypertarget{Exercise I.1.3}{}
\begin{problem}[1.3]
	$\vartriangleright$ Given a partition $\mathscr{P}$ on a set $S$, show how to define an equivalence 
	relation $\sim$ such that $\mathscr{P} = \mathscr{P}_{\sim}$. [\S1.5]
\end{problem}

\begin{solution}
Define a relation $\sim$ on $S$ as
$$a\sim b \iff  \exists A\in \mathscr{P}\text{, such that } a\in A, b\in A.$$
It is easy to prove this relation is an equivalence on $S$.
\end{solution}


% Problem 1.4
\begin{problem}[1.4]
	How many different equivalence relations can be defined on the set $\{1,2,3\}?$
\end{problem}

\begin{solution}
The number of equivalence relations is the same as the number of partitions. All possible partitions of $\{1,2,3\}$ are:
$$\{\{1\}, \{2\}, \{3\}\},$$
$$\{\{1\}, \{2,3\}\},$$
$$\{\{1,2\},\{3\}\},$$
$$\{\{1,3\}, \{2\}\},$$
$$\{\{1,2,3\}\}.$$
So, there are $5$ equivalence relations on $\{1,2,3\}$.
\end{solution}

% Problem 1.5
\begin{problem}[1.5]
	Give an example of a relation that is reflexive and symmetric but not
	transitive. What happens if you attempt to use this relation to define a
	partition on the set? (Hint: Thinking about the second question will help you
	answer the first one.)
\end{problem}

\begin{solution}
A relation $R$ on set $\{1,2,3\}$ is a subset  of $\{1,2,3\}\times \{1,2,3\}$. Taking subset 
$$\{\{1,1\}, \{2,2\}, \{3,3\}, \{1,2\},\{2,1\},\{1,3\}, \{3,1\}\}$$
It is easy to verify that $R$ is reflexive and symmetric. $1R2, 1R3$, but $2$ and $3$ are not related by $R$
because neither$\{2,3\}$ nor $\{3,2\}$ is in $R$. Thus, $R$ is not transitive.
\end{solution}

% Problem 1.6
\hypertarget{Exercise I.1.6}{}
\begin{problem}[1.6]
$\vartriangleright$ Define a relation $\sim$ on the set $\mathbb{R}$ of real numbers, by 
setting $a\sim b\iff b-a \in\mathbb{Z}$. 
Prove that this is an equivalence relation, and find a \textquoteleft compelling' description for $\mathbb{R}/\sim$. 
Do the same for the relation $\approx$ on the plane $\mathbb{R}\times\mathbb{R}$ defined by declaring 
$(a_1, a_2)\approx(b_1, b_2)\iff b_1-a_1 \in\mathbb{Z}$ and $b_2-a_2 \in\mathbb{Z}$. [\textsection II.8.1, II.8.10]
\end{problem}

\begin{solution}
(Reflexivity): $\forall a\in \mathbb{R}, a-a=0\in\mathbb{Z}$. It leads to the reflexivity.\\
(Symmetry): $\forall a, b\in \mathbb{R}$, if $a-b=k\in \mathbb{Z}$, then $b-a=-k\in \mathbb{Z}$. Thus, $\sim$ is symmetric.\\
(Transitivity): $\forall a, b,c\in \mathbb{R}$, if  $a-b=k_1\in  \mathbb{Z}, b-c=k_2\in  \mathbb{Z}$, 
then $a-c=k_1+k_2\in  \mathbb{Z}$. Therefore, $\sim$ is transitive.  
$$\mathbb{R}/\sim \cong [0,1).$$
The proof of $\approx$ being an equivalence relation $\mathbb{R}\times\mathbb{R}$ is similar to that of $\sim$. Thus, we omit it.
$$(\mathbb{R}\times\mathbb{R})/\approx \cong [0,1)\times [0,1).$$
\end{solution}

\subsection{\textsection2. Functions between sets}

% Problem 2.1
\hypertarget{Exercise I.2.1}{}
\begin{problem}[2.1]
$\vartriangleright$ How many different bijections are there between a set $S$ with $n$ elements
and itself? [\textsection II.2.1]
\end{problem}

\begin{solution}
The number of different bijections between $S$ with $n$ elements to itself is $n!$.
\end{solution}

% Problem 2.2
\hypertarget{Exercise I.2.2}{}
\begin{problem}[2.2]
 $\rhd$ Prove statement (2) in Proposition 2.1. You may assume that given a
  family of disjoint subsets of a set, there is a way to choose one element in
  each member of the family. [\S2.5, V3.3]
\end{problem}

\begin{solution}
($\Longrightarrow$): If $f: A\rightarrow B$ has right-inverse, then there is a function $g: B\rightarrow  A$ such that $f\circ g = \text{id}_{B}$. Thus
$$b = \text{id}_{B}(b) = f\circ g(b) = f(g(b)), \forall b\in B,$$that is, $f$ sends $g(b)\in A$ to $b$, showing $f$ is surjective. \\
($\Longleftarrow $): Now assume $f: A\rightarrow B$ is surjective. In order to construct a function
$g: B\rightarrow A$, we have to assign a unique $g(b) = a\in A$ for all $b\in B$. Since the fiber $f^{-1}(b)$, $\forall b\in B$ is not empty, 
we can select one element $a\in f^{-1}(b)$ as the image of $b$ under $g$. i.e.   
$$g(b) = a, \text{ where } a\in f^{-1}(b).$$
For this $g$, we have $f\circ g(b) = f(a) = b=\text{id}_{B}$. Thus $g$ is a right-inverse of $f$. 
\end{solution}

% Problem 2.3
\begin{problem}[2.3]
Prove that the inverse of a bijection is a bijection and that the composition
of two bijections is a bijection.
\end{problem}

\begin{solution}
If $f: A\rightarrow B$ is a bijection, then there exists $g: B\rightarrow A$ such that 
$$f\circ g = \text{id}_{B}, \indent g\circ f = \text{id}_{A}.$$
It implies $f$ is a left-inverse and a right-inverse of $g$. Therefore $g$ is a bijection.\\
If $f: A\rightarrow B$ and $g: B\rightarrow C$ are two bijections, then
$$f\circ f^{-1} = \text{id}_{B},\indent f^{-1}\circ f = \text{id}_{A},$$
$$g\circ g^{-1} = \text{id}_{C},\indent g^{-1}\circ g = \text{id}_{B}.$$
Thus,
$$(g\circ f)\circ(f^{-1}\circ g^{-1}) =(g\circ \text{id}_{B})\circ g^{-1}=g\circ g^{-1}= \text{id}_{C},$$
$$(f^{-1}\circ g^{-1})\circ(g\circ f) =f^{-1}\circ (\text{id}_{A}\circ f)=f^{-1}\circ f=\text{id}_{A}.$$
this means that $g\circ f$ has both left-inverse($f^{-1}\circ g^{-1}$) and right-inverse($f^{-1}\circ g^{-1}$). Therefore $g\circ f$ is a bijection.
\end{solution}

% Problem 2.4
\begin{problem}[2.4]
 $\rhd$ Prove that ‘isomorphism’ is an equivalence relation (on any set of sets).
[\S4.1]
\end{problem}

\begin{solution}
Suppose $S$ be a set of sets.\\
(Reflexivity): $\forall A\in S$, $A\cong A$ since $\text{id}_{A}: A\rightarrow A$ is the bijection between $A$ to itself.\\
(Symmetry): For $A\in S, B\in S$, if $A\cong B$, then there is a bijection $f: A\rightarrow B$. The inverse of $f$, denoted by $f^{-1}$
is a bijection between $B$ and $A$. This means $B\cong A$.\\
(Transitivity): For $A, B, C\in S$ satisfying $A\cong B, B\cong C$, there exist bijections $f: A\rightarrow B$ and $g: B\rightarrow C$. Composition of
bijections is bijection, i.e. $g\circ f: A\rightarrow C$ is a bijection. It implies $A\cong C$.
\end{solution}

% Problem 2.5
\begin{problem}[2.5]
  $\rhd$ Formulate a notion of \textit{epimorphism}, in the style
  of the notion of \textit{monomorphism} seen in \S 2.6, and prove a result
  analogous to Proposition 2.3, for epimorphisms and surjections.[\S 2.6, \S 4.2]
\end{problem}

\begin{solution}
 A function $f:A\to B$ is an \textit{epimorphism} if for all sets $Z$
  and all functions $\alpha',\alpha'':B\to Z$
  $$\alpha'\circ f = \alpha''\circ f\Longrightarrow \alpha' = \alpha''.$$\\
  Now, we prove: $f: A\rightarrow B$ is surjective if and only if $f$ is epimorphism.
  
  ($\implies$): If $f$ is surjective, then it has a right-inverse $g$ such that $$f\circ g = \text{id}_{B}.$$
  For all functions $\alpha',\alpha'': B\to Z$, 
  $$\alpha'\circ f = \alpha''\circ f \Longrightarrow \alpha'\circ (f\circ g) = \alpha''\circ (f\circ g) \Longrightarrow \alpha'\circ \text{id}_{B} = \alpha''\circ \text{id}_{B} \Longrightarrow \alpha' = \alpha''.$$
  
  ($\impliedby$): Let's take $Z=\{1,2,3\}$ and define 
  $$\alpha'=\begin{cases}
  			1, &b\in \text{im}f,\\
			2, &b\notin\text{im}f			
  			\end{cases}
  $$
    $$\alpha''=\begin{cases}
  			1, &b\in \text{im}f,\\
			3, &b\notin\text{im}f			
  			\end{cases}
  $$
 It is obvious that $\alpha'\circ f = \alpha''\circ f = 1$. Thus $\alpha' = \alpha''$. It means $\alpha'$ could not be $2$ and 
 $\alpha''$ could not be $3$. i.e. there isn't $b\notin \text{im}f$. It proves that $f$ is a surjection. 
\end{solution}

% Problem 2.6
\begin{problem}[2.6]
  With notation as in Example 2.4, explain how any function $f:A\to B$ determines
  a section of $\pi_A$.
\end{problem}

\begin{solution}
The section of a surjection is a right-inverse of the function. Define a function $g: A\rightarrow A\times B$ as
$g(a)=(a, f(a))$.
It's easy to verify 
$$\pi_{A} \circ g =\text{id}_A.$$ It proves that $g: A\rightarrow A\times B$ is a right-inverse of $\pi: A\times B \rightarrow A$.
\end{solution}

% Problem 2.7
\begin{problem}[2.7]
  Let $f:A\to B$ be any function. Prove that the graph $\Gamma_f$ of $f$ is
  isomorphic to $A$.
\end{problem}

\begin{solution}
Denote $\Gamma_f = \{(a, f(a)): a\in A\}$ and define $\pi_f: \Gamma_f \rightarrow A$ as 
$$\pi_f((a,f(a))) = a.$$
Let's define another functon $g: A\rightarrow \Gamma_f$ as $$g(a) = (a,f(a)).$$
So
$$(\pi_f\circ g)(a) = \pi_f(g(a)) = \pi_f((a,f(a)))=a \Rightarrow (\pi_f\circ g) = \text{id}_{A}.$$
and
$$(g\circ \pi_f)((a,f(a))) = g(\pi_f((a,f(a))))=g(a)=(a,f(a))\Rightarrow (g\circ \pi_f) = \text{id}_{\Gamma_f}.$$
This means $\pi_f: \Gamma_f\rightarrow A$ is a bijection. i.e. $\Gamma_f\cong A$.
\end{solution}

% Problem 2.8.
\begin{problem}[2.8]
  Describe as explicitly as you can all terms in the canonical decomposition (cf.
  \S2.8) of the function $\mathbb{R}\to\mathbb{C}$ defined by $r\mapsto e^{2\pi
    ir}$. (This exercise matches one previously. Which one?)
\end{problem}

\begin{solution}
Let $f: \mathbb{R}\to\mathbb{C}$ be $f(r) = e^{2\pi ir}$. 
$$f(r_1) = f(r_2) \iff 1-e^{2\pi i(r_2-r_1)} = 0 \iff r_2-r_1\in\mathbb{Z}.$$
Denote the unit circle $\{e^{2\pi i r} | r\in\mathbb{R}\}$ on the complex plane as $\mathbb{C}_{|z|=1}$.
Define an equivalence relation $\sim$ on $\mathbb{R}$ as $a\sim b \iff a-b\in\mathbb{Z}$ and 
a function $\tilde{f}: \mathbb{R}/\sim\rightarrow \mathbb{C}_{|z|=1}$ as
$$\tilde{f}([r]_{\sim}) := e^{2\pi i r}.$$
It is easy to verify that $\tilde{f}$ is well-defined and is a bijection and the following diagram holds
\[ 
\begin{tikzcd}
\mathbb{R} \arrow[r, "f"] \arrow[d, "\pi"'] &[0.5em]  \mathbb{C}  \\
 \mathbb{R}/\sim \arrow[r, "\tilde{f}"'] & \mathbb{C}_{|z|=1} \arrow[hookrightarrow,u, "i"] 
\end{tikzcd}
\]
where $\pi$ is the natural projection, $\textit{i}$ is the inclusion. Therefore, $f=\textit{i}\circ\tilde{f}\circ\pi$.
\end{solution}


% Problem 2.9
\hypertarget{Exercise I.2.9}{}
\begin{problem}[2.9]
  $\rhd$ Show that if $A'\cong A''$ and $B'\cong B''$, and further
  $A'\cap B'=\varnothing$ and $A''\cap B''=\varnothing$, then $A'\cup B'\cong A''\cup
  B''$. Conclude that the operation $A\amalg B$ is well-defined up to
  \textit{isomorphism} (cf. \S2.9) [\S2.9, 5.7]
\end{problem}

\begin{solution}
Let $f_1: A'\to A'', f_2: B'\to B''$ be the bijections. Define 
$$f(x)=\begin{cases}
	f_1(x), &x\in A',\\
	f_2(x), &x\in B' 	
\end{cases}$$
Since $A'\cap B'=\varnothing$, the function $f$ is well-defined. Now, we prove that $f$ is a bijection.

(Injection): For $x_1, x_2\in A'\cup B'$ and $x_1\neq x_2$, there are four cases:
\begin{enumerate}
\item if $x_1\in A', x_2\in A'$, then $f(x_1)=f_1(x_1)\neq f_1(x_2) = f(x_2)$.
\item if $x_1\in B', x_2\in B'$, then $f(x_1)=f_2(x_1)\neq f_2(x_2) = f(x_2)$.
\item if $x_1\in A', x_2\in B'$, then $f(x_1)=f_1(x_1)\in A'', f(x_2)=f_2(x_2)\in B"$. Since $A''\cap B'' = \varnothing$, $f(x_1)\neq f(x_2)$.
\item if $x_1\in B', x_2\in A'$, then $f(x_1)=f_2(x_1)\in B'', f(x_2)=f_1(x_2)\in A"$. Since $A''\cap B'' = \varnothing$, $f(x_1)\neq f(x_2)$.
\end{enumerate}
It proves that $f$ is an injection.


(Surjection): For $x\in A''\cup B"$, we have either $x\in A''$ or $x\in B"$. Therefore there exists $a\in A'$ or $a\in B'$ such that $f(a)=f_1(a)=x$ or $f(a)=f_2(a)=x$.

The existence of a bijection between $A'\cup B'$ and $A''\cup B''$ implies $A'\cup B'\cong A''\cup B''$.

Let $A', A''$ be two 'copies' of $A$ and $B', B''$ be two 'copies' of $B$. It is obvious that $A'\cong A''$ and $B'\cong B''$. 
Meanwhile, by the disjoin union, we know that $A'\cap B'=\varnothing$ and $A''\cap B''=\varnothing$. By the result of this problem, we have 
$$A'\cup B'\cong A''\cup B''.$$ It implies $A\coprod B$ is well-defined in term of isomorphism.   
\end{solution}

% Problem 2.10
\hypertarget{Exercise I.2.10}{}
\begin{problem}[2.10]
  $\rhd$ Show that if $A$ and $B$ are finite sets, then $\abs{B^A} =
  \abs{B}^{\abs{A}}$. [\S2.1, 2.11, I.4.1]
\end{problem}

\begin{solution}
By definition, $B^A = \{f | f: A\to B\}$. If both $A$ and $B$ are finite, then for each $x\in A$, it can be assigned $|B|$ different values. 
Therefore, the number of functions from $A$ to $B$ is $\abs{B}^{\abs{A}}$. i.e. $\abs{B^A} =
  \abs{B}^{\abs{A}}$
\end{solution}

% Problem 2.11
\begin{problem}[2.11]
  $\rhd$ In view of Exercise 2.10, it is not unreasonable to use $2^A$ to denote
  the set of functions from an arbitrary set $A$ to a set with $2$ elements (say
  $\{0,1\}$). Prove that there is a bijection between $2^A$ and the \textit{power
    set} of $A$ (cf. \S1.2). [\S1.2, III.2.3]
\end{problem}

\begin{solution}
Let $\mathcal{P}(A)$ be the power set of $A$. Define a mapping between $\mathcal{P}(A)\to 2^A$ as following:
$$f(X) = g, \text{ where } g(X) = 1, g(A\backslash X) = 0.$$ Let's prove this map is a bijection.

(Injection): For $X_1, X_2\in \mathcal{P}(A), X_1\neq X_2$, we have two functions 
$$g_1\big{|}_{X_1} = 1, g_1\big{|}_{A\backslash X_1} = 0,$$
$$g_2\big{|}_{X_2} = 1, g_1\big{|}_{A\backslash X_2} = 0.$$
If $X_1\backslash X_2\neq \varnothing$, then $g_1\big{|}_{X_1\backslash X_2} = 1\neq 0 = g_2\big{|}_{X_1\backslash X_2}$.
If $X_2\backslash X_1\neq \varnothing$, then $g_1\big{|}_{X_2\backslash X_1} = 0\neq 1 = g_2\big{|}_{X_2\backslash X_1}$. 
It proves that $g_1 \neq g_2$.

(Surjection): For a function $g: A\to \{0,1\}$, $$f(\{x| x\in A \text{ and } g(x)= 1\}) = g,$$
showing $f$ is surjection.
\end{solution}


\subsection{\textsection3. Categories}

% Problem 3.1
\hypertarget{Exercise I.3.1}{}
\begin{problem}[3.1]
  $\rhd$ Let $\mathsf{C}$ be a category. Consider a structure $\mathsf{C}^{op}$ with:
  \begin{itemize}
  \item $\Obj(\mathsf{C}^{op}) := \Obj(\mathsf{C})$;
  \item for $A$, $B$ objects of $\mathsf{C}^{op}$ (hence, objects of $\mathsf{C}$), $\Hom_{\mathsf{C}^{op}} (A,B) := \Hom_\mathsf{C}(B,A)$
  \end{itemize}
  Show how to make this into a category (that is, define composition of morphisms
  in $\mathsf{C}^{op}$ and verify the properties listed in \textsection3.1).
  Intuitively, the `opposite' category $\mathsf{C}^{op}$ is simply obtained by `reversing all the
  arrows' in C. [5.1, \textsection VIII.1.1, \textsection IX.1.2, IX.1.10]
\end{problem}

\begin{solution}
Let's define $\ast$ as the composition of morphisms between objects in $\mathsf{C}^{op}$ and $\circ$ 
as the composition of morphisms between objects in $\mathsf{C}$.
\begin{enumerate}
\item For every object $A$ of $\Obj(\mathsf{C}^{op})$, the identity $1_A \in \Hom_{\mathsf{C}^{op}} (A,A)$ is the one 
in $\Hom_{\mathsf{C}}(A,A)$.
\item For $f\in \Hom_{\mathsf{C}^{op}} (A,B)$ and $g\in \Hom_{\mathsf{C}^{op}}(B,C)$. 
$g\circ f \in \Hom_\mathsf{C}(C,A)$. By definition, $g\circ f \in \Hom_{\mathsf{C}^{op}}(A,C)$. i.e. The composition
$$(f,g) \mapsto f\ast g := g\circ f$$ determines a morphism in $\Hom_{\mathsf{C}^{op}}(A,C)$.
\item For $f\in \Hom_{\mathsf{C}^{op}} (A,B)$, $g\in \Hom_{\mathsf{C}^{op}}(B,C)$ and $g\in \Hom_{\mathsf{C}^{op}}(C,D)$, then
$$(f\ast g)\ast h = h\circ (g\circ f) = (h\circ g)\circ f = f\ast(g\ast h).$$
i.e. The composition laws is associative.
\item For $1_A \in \Hom_{\mathsf{C}^{op}}(A,A)$, $1_B \in \Hom_{\mathsf{C}^{op}}(B,B)$ and $f\in \Hom_{\mathsf{C}^{op}} (A,B)$, we have
$$f\ast 1_A = 1_A\circ f = f, \text{ and } 1_B\ast f = f\circ 1_B = f.$$
i.e. The identity morphisms are identities with respect to composition.
\end{enumerate}
It proves the structure $\mathsf{C}^{op}$ defined in this problem is a category. 
\end{solution}

% Problem 3.2
\begin{problem}[3.2]
  If $A$ is a finite set, how large is $\mathrm{End}_{\mathsf{Set}}(A)$?
\end{problem}

\begin{solution}
The set $\mathrm{End}_{\mathsf{Set}}(A)$ is the set of functions (as morphism) from $A$ to $A$. Let $\abs{A}=n$, then 
$\abs{\mathrm{End}_{\mathsf{Set}}(A)} = n^n$.
\end{solution}

% Problem 3.3
\begin{problem}[3.3]
  $\vartriangleright$ Formulate precisely what it means to say that $1_a$ is an identity with respect to composition in Example 3.3, 
  and prove this assertion. [\textsection3.2]
\end{problem}

\begin{solution}
$1_a\in \Hom(a,a)$ tells us $a\sim a$, $1_b\in \Hom(b,b)$ tells us $b\sim b$. Let $f\in\Hom(a,b)$. It implies $a\sim b$.
$$a\sim a \text{ and } a\sim b \Longrightarrow  a\sim b => f1_a = f,$$
$$a\sim b \text{ and } b\sim b \Longrightarrow  a\sim b => 1_bf = f.$$
It proves that identity morphisms are identity with respect to composition.
\end{solution}

% Problem 3.4
\begin{problem}[3.4]
  Can we define a category in the style of Example 3.3 using the relation $<$ on
  the set $\mathbb{Z}$?
\end{problem}

\begin{solution}
No. We could not have an identity in $\Hom(n,n)$ since $n \nless n$.
\end{solution}

% Problem 3.5
\begin{problem}[3.5]
  $\rhd$ Explain in what sense Example 3.4 is an instance of the categories
  considered in Example 3.3. [\S 3.2]
\end{problem}

\begin{solution}
The set inclusion $\subseteq$ is a relation defined on members in $\mathscr{P}$.
\begin{enumerate}
\item (Reflexivity): For any $A\in \mathscr{P}$, $A\subseteq A$.
\item (Transitivity): For $A, B, C\in \mathscr{P}$, if $A\subseteq B, B\subseteq C$, then $A\subseteq C$.
\end{enumerate} 
Thus, the power set $\mathscr{P}$ and set inclusion satisfy the assumption of Example 3.3. Therefore it is a category.
\end{solution}

% Problem 3.6
\begin{problem}[3.6]
  $\rhd$ (Assuming some familiarity with linear algebra.) Define a category
  $\mathsf{V}$ by taking $\Obj(\mathsf{V}) = \mathbb{N}$ and letting
  $\Hom_{\mathsf{V}}(m,n) = $ the set of $m\times n$ matrices with real
  entries, for all $m,n\in\mathbb{N}$. (We will leave the reader the task of
  making sense of a matrix with 0 rows or columns.) Use product of matrices to
  define composition. Does this category `feel' familiar? [\S VI.2.1, \S VIII.1.3]
\end{problem}

\begin{solution}

\end{solution}


% Problem 3.7
\begin{problem}[3.7]
  $\rhd$ Define carefully the objects and morphisms in Example 3.7, and draw the
  diagram corresponding to compositon. [\S 3.2]
\end{problem}

\begin{solution}
Let $\mathbb{C}$ be a category. We define the structure, denoted by $\mathsf{C}^{A}$ as following:
\begin{enumerate}
\item $\Obj(\mathsf{C}^{A})$ = all morphisms from $A$ to any object of $\mathsf{C}$. i.e. an object
of $\mathsf{C}^{A}$ is a morphism $f\in \Hom_{\mathsf{C}}(A, Z)$ for some object $Z$ of $\mathsf{C}$.
\item For $f\in \Hom_{\mathsf{C}}(A, Z)$ and $g\in \Hom_{\mathsf{C}}(A, X)$, 
define $$\Hom_{\mathsf{C}^{A}}(f,g) = \{\sigma\big{|}\sigma\in \Hom_{\mathsf{C}}(Z, X)\text{ such that } \sigma f = g\}.$$
\end{enumerate}
Equipped the structure defined above, $\mathsf{C}^{A}$ is a category. In fact
\begin{enumerate}
\item For object $f\in \Hom_{\mathsf{C}}(A, Z)$ of $\mathsf{C}^{A}$, the identity $1_{f}$ can be chosen as the 
identity $1_{Z}$ in $\Hom_{\mathsf{C}}(Z, Z)$ since $f = 1_Zf$.
\item Let $f\in \Hom_{\mathsf{C}}(A, Z_1),g\in \Hom_{\mathsf{C}}(A, Z_2),h\in \Hom_{\mathsf{C}}(A, Z_3)$ be objects 
of $\mathsf{C}^{A}$.
$$f_{\mathsf{C}^{A}}\in \Hom_{\mathsf{C}^{A}}(f, g)\iff f_{\mathsf{C}^{A}}(f) = g 
\iff \exists \sigma_1\in \Hom_{\mathsf{C}}(Z_1, Z_2) \text{ s.t. } \sigma_1 f = g,$$ 
$$g_{\mathsf{C}^{A}}\in \Hom_{\mathsf{C}^{A}}(g, h) \iff g_{\mathsf{C}^{A}}(g) = h 
\iff \exists \sigma_2\in \Hom_{\mathsf{C}}(Z_2, Z_3) \text{ s.t. } \sigma_2 g = h.$$

Define the composition law $\ast$ as 
\begin{equation*}
\begin{split}(f_{\mathsf{C}^{A}} \ast g_{\mathsf{C}^{A}})(f) &:= g_{\mathsf{C}^{A}}(f_{\mathsf{C}^{A}}(f)) \\
&=g_{\mathsf{C}^{A}}(\sigma_1 f)\\
&=\sigma_2(\sigma_1 f) \\
&=(\sigma_2\sigma_2)f
\end{split}
\end{equation*}
Since $\mathsf{C}$ is a category, $\sigma_2\sigma_2 \in \Hom_{\mathsf{C}}(Z_1, Z_3)$. By definition, $\sigma_2\sigma_1$ defines 
a morphism from $f$ to $h$.
\item For $f_{\mathsf{C}^{A}}\in \Hom_{\mathsf{C}^{A}}(f, g), g_{\mathsf{C}^{A}}\in \Hom_{\mathsf{C}^{A}}(g, h), 
h_{\mathsf{C}^{A}}\in \Hom_{\mathsf{C}^{A}}(h, l)$, we have
\begin{equation*}
\begin{split}
((f_{\mathsf{C}^{A}}\ast g_{\mathsf{C}^{A}})\ast h_{\mathsf{C}^{A}})(f) &= h_{\mathsf{C}^{A}}((f_{\mathsf{C}^{A}}\ast g_{\mathsf{C}^{A}})(f))\\
&=h_{\mathsf{C}^{A}}(g_{\mathsf{C}^{A}}(f_{\mathsf{C}^{A}}(f))) \\&= (\sigma_h\sigma_g\sigma_f)(f)\\
&=(g_{\mathsf{C}^{A}}\ast h_{\mathsf{C}^{A}})(f_{\mathsf{C}^{A}}(f))\\
&=(f_{\mathsf{C}^{A}}\ast (g_{\mathsf{C}^{A}}\ast h_{\mathsf{C}^{A}}))(f)
\end{split}
\end{equation*}
i.e the composition law is associative.
\item The proof of $$1_{\mathsf{C}^{A}}\ast f_{\mathsf{C}^{A}}= f_{\mathsf{C}^{A}}, f_{\mathsf{C}^{A}}\ast 1_{\mathsf{C}^{A}} = f_{\mathsf{C}^{A}}$$ is straightforward.
\end{enumerate}
This proves that $\mathsf{C}^{A}$ with structure defined in this problem is a category.
\end{solution}

% Problem 3.8
\begin{problem}[3.8]
  $\rhd$ A \textit{subcategory} ${\mathsf{C}'}$ of a category $\mathsf{C}$ consists of a
  collection of objects of $\mathsf{C}$, with morphisms
  $\Hom_{\mathsf{C}'}(A,B) \subseteq \Hom_\mathsf{C}(A,B)$ for all objects $A,B\in\Obj({\mathsf{C}'})$, such
  that identities and compositions in $\mathsf{C}$ make ${\mathsf{C}'}$ into a category. A
  subcategory ${\mathsf{C}'}$ is \textit{full} if $\Hom_{\mathsf{C}'}(A,B) = \Hom_\mathsf{C}(A,B)$ for all
  $A,B\in\Obj({\mathsf{C}'})$. Construct a category of \textit{infinite sets} and explain
  how it may be viewed as a full subcategory of $\mathsf{Set}$. [4.4,\S VI.1.1, \S
  VIII.1.3]
\end{problem}

\begin{solution}
Denote the collection of infinite sets as $\mathsf{InfSet}$ and define the following structure on it
\begin{enumerate}
\item $\Obj(\mathsf{InfSet})=$ the class of sets with infinitely many elements;
\item for $A, B\in \Obj(\mathsf{InfSet})$, $\Hom_{\mathsf{InfSet}}(A,B) = \{f \big{|} f: A\to B \text{ is a function from } A \text{ to } B\}$.
\end{enumerate}
It is easy to prove that $\mathsf{InfSet}$ is a category and it is a full subcategory of $\mathsf{Set}$.
\end{solution}

% Problem 3.9
\hypertarget{Exercise I.3.9}{}
\begin{problem}[3.9]
  $\rhd$ An alternative to the notion of \textit{multiset} introduced in
  \S2.2 is obtained by considering sets endowed with equivalence relations;
  equivalent elements are taken to be multiple instance of elements `of the same
  kind'. Define a notion of morphism between such enhanced sets, obtaining a
  category ${\mathsf{MSet}}$ containing (a `copy' of) $\Set$ as a full subcategory. (There
  may be more than one reasonable way to do this! This is intentionally an
  open-ended exercise.) Which objects in $\mathsf{MSet}$ determine ordinary multisets as
  defined in \S2.2 and how? Spell out what a morphism of multisets would be from
  this point of view. (There are several natural motions of morphisms of
  multisets. Try to define morphisms in $\mathsf{MSet}$ so that the notion you obtain for
  ordinary multisets captures your intuitive understanding of these objects.)
  [\S2.2, \S3.2, 4.5]
\end{problem}

% Problem 3.10
\begin{problem}[3.10]
  Since the objects of a category $\mathsf{C}$ are not (necessarily) sets, it is not clear
  how to make sense of a notion of `subobject' in general. In some situations it
  \textit{does} make sense to talk about subobjects, and the subobjects of any
  given object $A$ in $\mathsf{C}$ are in one-to-one correspondence with the morphisms
  $A\to\Omega$ for a fixed, special object $\Omega$ of $\mathsf{C}$, called a
  \textit{subobject classifier}. Show that $\Set$ has a subobject classifier.
\end{problem}

\begin{solution}
\end{solution}

% Problem 3.11
\begin{problem}[3.11]
  $\rhd$ Draw the relevant diagrams and define composition and identities for the
  category $\mathsf{C}^{A,B}$ mentioned in Example 3.9. Do the same for the category
  $\mathsf{C}^{\alpha,\beta}$ mentioned in Example 3.10. [\S5.5, 5.12]
\end{problem}

\begin{solution}
Let $\mathsf{C}$ be a category, we define the following structure, denoted as $\mathsf{C}^{A,B}$:
\begin{enumerate}
\item $\Obj({\mathsf{C}^{A,B}})=\{(f,g, Z)\big{|} 
\forall Z\in \mathsf{C}, f\in \Hom_{\mathsf{C}}(A, Z), g\in \Hom_{\mathsf{C}}(A, Z)\}$
%
\[\begin{tikzcd}
A \arrow[dr, "f"] & \\
& Z \\
B \arrow[ur, "g"'] &
\end{tikzcd}
\]
%

\item A morphism between triple $(f_1, g_1, Z_1)$ and $(f_2, g_2, Z_2)$, denoted by 
$\sigma_{[(f_1,g_1,Z_1),(f_2,g_2,Z_2)]}$, 
is defined as
\begin{equation*}
\begin{split}
&\sigma_{[(f_1,g_1,Z_1),(f_2,g_2,Z_2)]}((f_1,g_1,Z_1)) = (f_2,g_2,Z_2) \iff \\
&\exists \sigma\in \Hom_{\mathsf{C}}(Z_1, Z_2) \text{ s.t. } f_2=\sigma f_1, g_2 = \sigma g_1.
\end{split}
\end{equation*}
%
\[
\begin{tikzcd}
A \arrow[dr, "f_1"] \arrow[drr, bend left, "f_2" near end] & \\
& Z_1 \arrow[r, "\sigma"] & Z_2 \\
B \arrow[ur, "g_1"'] \arrow[urr, bend right, "g_2"' near end]&
\end{tikzcd}
\]
%

\item The composition law of morphisms, denoted by $\ast$, is defined as
\begin{equation*}
\begin{split}
&(\sigma_{[(f_1,g_1,Z_1),(f_2,g_2,Z_2)]}\ast \sigma_{[(f_2,g_2,Z_2),(f_3,g_3,Z_3)]})((f_1,g_1,Z_1))\\=&
\sigma_{[(f_2,g_2,Z_2),(f_3,g_3,Z_3)]}(\sigma_{[(f_1,g_1,Z_1),(f_2,g_2,Z_2)]}((f_1,g_1,Z_1)))\\=&
\sigma_{[(f_2,g_2,Z_2),(f_3,g_3,Z_3)]}((\sigma_1f_1,\sigma_1 g_1, Z_2))\\=&
(\sigma_2\sigma_1f_1, \sigma_2\sigma_1g_1,Z_3)
\end{split}
\end{equation*}
%
\[
\begin{tikzcd}
A \arrow[dr, "f_1"] \arrow[drr, bend left, "f_2" near end] \arrow[drrr, bend left, "f_3" near end] & & \\
& Z_1 \arrow[r, "\sigma_1"] & Z_2 \arrow[r, "\sigma_2"] & Z_3 \\
B \arrow[ur, "g_1"'] \arrow[urr, bend right, "g_2"' near end] \arrow[urrr, bend right, "g_3"' near end] & &
\end{tikzcd}
\begin{tikzcd}
A \arrow[dr, "f_1"] \arrow[drr, bend left, "f_3" near end] & \\
& Z_1 \arrow[r, "\sigma_2\sigma_1"] & Z_3 \\
B \arrow[ur, "g_1"'] \arrow[urr, bend right, "g_3"' near end] &
\end{tikzcd}
\]
%
\end{enumerate}

For category $\mathsf{C}^{\alpha,\beta}$
\begin{enumerate}
\item $\Obj(\mathsf{C}^{\alpha,\beta})=$commutative diagrams
%
\[
\begin{tikzcd}
& A \arrow[dr, "f"]& \\
C \arrow[ur, "\alpha"] \arrow[dr, "\beta"'] & & Z\\
& B \arrow[ur, "g"'] &
\end{tikzcd}
\]
%
\item morphisms
%
\[
\begin{tikzcd}
& A \arrow[dr, "f_1"] \arrow[drr, bend left, "f_2" near end] & & \\
C \arrow[ur, "\alpha"] \arrow[dr, "\beta"] & & Z_1\arrow[r, "\sigma"] & Z_2\\
& B \arrow[ur, "g_1"'] \arrow[urr, bend right, "g_2"' near end] & &
\end{tikzcd}
\]
%
\item composition law of morphisms
%
\[
\begin{tikzcd}
& A \arrow[dr, "f_1"] \arrow[drr, bend left, "f_2" near end] \arrow[drrr, bend left, "f_3" near end] &  & & \\
C \arrow[ur, "\alpha"] \arrow[dr, "\beta"] & & Z_1\arrow[r, "\sigma_1"] & Z_2 \arrow[r, "\sigma_2"] & Z_3\\
& B \arrow[ur, "g_1"'] \arrow[urr, bend right, "g_2"' near end] \arrow[urrr, bend right, "g_3"' near end]& & &
\end{tikzcd}
\begin{tikzcd}
& A \arrow[dr, "f_1"] \arrow[drr, bend left, "f_3" near end] & & \\
C \arrow[ur, "\alpha"] \arrow[dr, "\beta"] & & Z_1\arrow[r, "\sigma_2\sigma_1"] & Z_3\\
& B \arrow[ur, "g_1"'] \arrow[urr, bend right, "g_3"' near end] & &
\end{tikzcd}
\]
%
\end{enumerate}
This proves that $\mathsf{C}^{\alpha,\beta}$ with the structure defined above is a category.
\end{solution}
