\section{Chapter I.\hspace{0.2em} Preliminaries: Set theory and categories}

\subsection{\textsection1. Naive Set Theory}

% Problem 1.1
\begin{problem}[1.1]
	Locate a discussion of Russel's paradox, and understand it.
\end{problem}
\begin{solution}
\end{solution}

% Problem 1.2
\hypertarget{Exercise I.1.2}{}
\begin{problem}[1.2]
	$\vartriangleright$ Prove that if $\sim$ is an equivalence relation on a set $S$, then
	the corresponding family $\mathscr{P}_{\sim}$ defined in \S1.5 is indeed a
	partition of $S$; that is, its elements are nonempty, disjoint, and their union
	is $S$. [\S1.5]
\end{problem}

\begin{solution}
\end{solution}

% Problem 1.3
\hypertarget{Exercise I.1.3}{}
\begin{problem}[1.3]
	$\vartriangleright$ Given a partition $\mathscr{P}$ on a set $S$, show how to define an equivalence relation $\sim$ such that $\mathscr{P} = \mathscr{P}_{\sim}$. [\S1.5]
\end{problem}

\begin{solution}
	
\end{solution}


% Problem 1.4
\begin{problem}[1.4]
	How many different equivalence relations can be defined on the set $\{1,2,3\}?$
\end{problem}

\begin{solution}
\end{solution}


% Problem 1.5
\begin{problem}[1.5]
	Give an example of a relation that is reflexive and symmetric but not
	transitive. What happens if you attempt to use this relation to define a
	partition on the set? (Hint: Thinking about the second question will help you
	answer the first one.)
\end{problem}

\begin{solution}
\end{solution}

% Problem 1.6
\hypertarget{Exercise I.1.6}{}
\begin{problem}[1.6]
Define a relation $\sim$ on the set $\mathbb{R}$ of real numbers, by setting $a\sim b\iff b-a \in\mathbb{Z}$. Prove that this is an equivalence relation, and find a \textquoteleft compelling' description for $\mathbb{R}/\sim$. Do the same for the relation $\approx$ on the plane $\mathbb{R}\times\mathbb{R}$ defined by declaring $(a_1, a_2)\approx(b_1, b_2)\iff b_1-a_1 \in\mathbb{Z}$ and $b_2-a_2 \in\mathbb{Z}$. [\textsection II.8.1, II.8.10]
\end{problem}
\begin{solution}
\end{solution}
